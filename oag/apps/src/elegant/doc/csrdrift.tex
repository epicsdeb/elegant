This element has a number of models for simulation of CSR in drift
spaces following CSRCSBEND elements.  Note that all models allow
support splitting the drift into multiple CSRDRIFT elements.
One can also have intervening elements like quadrupoles,
as often happens in chicanes.  The CSR effects inside such
intervening elements are applied in the CSRDRIFT downstream of
the element.

For a discussion of some of the methods behind this element, see
M. Borland, ``Simple method for particle tracking with coherent
synchrotron radiation,'' Phys. Rev. ST Accel. Beams 4, 070701 (2001).

{\bf N.B.}: by default, this element uses 1 CSR kick (N\_KICKS=1) at the
center of the drift.  This is usually not a good choice.  I usually
use the DZ parameter instead of N\_KICKS, and set it to something
like 0.01 (meters).  The user should vary this parameter to assess
how small it needs to be.

The models are as following, in order of decreasing sophistication and accuracy:
\begin{itemize}
\item G. Stupakov's extension of Saldin et al.  Set USE\_STUPAKOV=1.
The most advanced model at present is based on a private communication
from G. Stupakov (SLAC), which extends equation 87 of the one-dimensional
treatment of Saldin et al. (NIM A 398 (1997) 373-394) to include the
post-dipole region.  This model includes not only the attenuation of the
CSR as one proceeds along the drift, but also the change in the shape of
the ``wake.''

This model has the most sophisticated treatment for intervening
elements of any of the models.  For example, if you have a sequence
{\tt CSRCSBEND}-{\tt CSRDRIFT}-{\tt CSRDRIFT} and compare it with the
sequence {\tt CSRCSBEND}-{\tt CSRDRIFT}-{\tt DRIFT} -{\tt CSRDRIFT},
keeping the total drift length constant, you'll find no change in the
CSR-induced energy modulation.  The model back-propagates to the
beginning of the intervening element and performs the CSR computations
starting from there.

This is the slowest model to run.  It uses the same binning and
smoothing parameters as the upstream CSRCSBEND.  If run time is a
problem, compare it to the other models and use only if you get
different answers.

\item M. Borland's model based on Saldin et al. equations 53 and 54.
Set USE\_SALDIN54=1.  This model computes the fall-off of the CSR wake
from the work of Saldin and coworkers, as described in the reference
above.  It does not compute the change in the shape of the wake.  The
fall-off is computed approximately as well, based on the fall-off for
a rectangular current distribution.  The length of this rectangular
bunch is taken to be twice the bunch length computed according to the
BUNCHLENGTH\_MODE parameter (see below).  If your bunch is nearly
rectangular, then you probably want BUNCHLENGTH\_MODE of
``90-percentile''.

\item Exponential attenuation of a CSR wake with unchanging shape.
There are two options here.  First, you can provide the attenuation
length yourself, using the ATTENUATION\_LENGTH parameter.  Second, you
can set USE\_OVERTAKING\_LENGTH=1 and let {\tt elegant} compute the
overtaking length for use as the attenuation length.  In addition, you
can multiply this result by a factor if you wish, using the
OL\_MULTIPLIER parameter.

\item Beam-spreading model.  This model is not recommended.  It is
based on the seemingly plausible idea that CSR spreads out just like
any synchrotron radiation, thus decreasing the intensity.  The model
doesn't reproduce experiments.

\end{itemize}

The ``Saldin 54'' and ``overtaking-length'' models rely on computation
of the bunch length, which is controlled with the BUNCHLENGTH\_MODE
parameter.  Nominally, one should use the true RMS, but when the beam
has temporal spikes, it isn't always clear that this is the best
choice.  The choices are ``rms'', ``68-percentile'', and
``90-percentile''.  The last two imply using half the length
determined from the given percentile in place of the rms bunch length.
I usually use 68-percentile, which is the default.

