The input file for this element uses a simple text format.  It is identical
to the output in the {\tt printout} file generated by the {tt matrix\_output}
command.  For example, for a 1st-order matrix, the file would have the
following appearance:\\
{\em description}: {\em C1 C2 C3 C4 C5 C6}\\
R1: {\em R11 R12 R13 R14 R15 R16}\\
R2: {\em R21 R22 R23 R24 R25 R26}\\
R3: {\em R31 R32 R33 R34 R35 R36}\\
R4: {\em R41 R42 R43 R44 R45 R46}\\
R5: {\em R51 R52 R53 R54 R55 R56}\\
R6: {\em R61 R62 R63 R64 R65 R66}\\

Items in normal type must be entered exactly as shown, whereas those in
italics must be provided by the user.  The colons are important!
For this particular example, one would set {\tt ORDER=1} in the {\tt MATR}
definition.  In general, the {\em Ci} are zero, except for {\em C5}, which
is usually equal to the length of the element (which must be specified with
the {\tt L} parameter in the {\tt MATR} definition).