This element allows simulation of a transverse impedance using a
``broad-band'' resonator or an impedance function specified in a file.
The impedance is defined as the Fourier transform of the wake function
\begin{equation}
Z(\omega) = \int_{-\infty}^{+\infty} e^{-i \omega t} W(t) dt
\end{equation}
where $i = \sqrt{-1}$, $W(t)=0 for t<0$, and $W(t)$ has units of
$V/C/m$.  Note that there is no factor of $i$ in front of the
integral.  Thus, in {\tt elegant} the transverse impedance is simply
the Fourier transform of the wake.  This makes it easy to convert data
from a program like ABCI into the wake formalism using {\tt sddsfft}.

For a resonator impedance, the functional form is
\begin{equation}
Z(\omega) = \frac{-i\omega_r}{\omega} \frac{R_s}{1 + iQ(\frac{\omega}{\omega_r} - \frac{\omega_r}{\omega})},
\end{equation}
where $R_s$ is the shunt impedance in $Ohms/m$, $Q$ is the quality
factor, and $\omega_r$ is the resonant frequency.

When providing an impedance in a file, the user must be careful to conform to these
conventions.

Other notes:
\begin{enumerate}
\item The frequency data required from the input file is {\em not} $\omega$, but rather
  $f = \omega/(2 \pi)$.
\item The default smoothing setting ({\tt SG\_HALFWIDTH=4}), may apply too much smoothing.
It is recommended that the user vary this parameter if smoothing is employed.
\end{enumerate}

