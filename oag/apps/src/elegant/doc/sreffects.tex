This element provides an alternative to element-by-element modeling of synchrotron radiation.
It should be used with care because the results will not necessarily be self-consistent.
This is particularly an issue when there is dispersion at the location of the \verb|SREFFECTS|
element.

There are several instances in which one may want to use this element:
\begin{itemize}
\item Simulation of instabilities or other dynamics where radiation damping or quantum excitation is important.
\item Simulation of dynamics with an rf cavity when the synchronous phase is significantly
  different from 180 degrees, so that average radiation losses must be included.
\item Computation of dynamic and momentum aperture in the presence of radiation damping.
\end{itemize}

The major parameters (\verb|JX|, \verb|JY|, \verb|EXREF|, \verb|SDELTAREF|,
\verb|DDELTAREF|, and \verb|PREF|) can be supplied explicitly by the user, or filled in by {\tt elegant}
if the \verb|twiss_output| command is given with \verb|radiation_integrals=1|.

In explicit initialization, the user supplies the quantities {\tt
EXREF}, {\tt EYREF}, {\tt SDELTAREF}, {\tt DDELTAREF}, and {\tt
PREF}.  These are, respectively, the reference values for the x-plane
emittance, y-plane emittance, fractional momentum spread, energy loss
per turn, and momentum.  The first four values pertain to the
reference momentum.  {\tt JX}, {\tt JY}, and {\tt JDELTA} may also
be given, although the defaults work for typical lattices.

In automatic initialization, the user turns on the radiation integral
feature in {\tt twiss\_output}, causing {\tt elegant} to automatically
compute the above quantities.  This will occur only if {\tt PREF=0}.
The {\tt COUPLING} parameter can be used to change the partitioning of
quantum excitation between the horizontal and vertical planes.
Because the radiation integrals computation in \verb|twiss_output| pertains to the
horizontal plane only, the user must supply either \verb|EYREF| or \verb|COUPLING| if
non-zero vertical emittance is desired.

The user may elect to turn off some aspects of the synchrotron radiation model.  These should be
changed from the default values with care!
\begin{itemize}
\item \verb|DAMPING| --- Default is 1.  If set to 0, then no radiation damping effects will be included.
  More precisely, it is equivalent to setting \verb|JX=JY=JDELTA=1|.  Damping still occurs at any
  rf cavities (since {\tt elegant} works in trace space).
\item \verb|QEXCITATION| --- Default is 1.  If set to 0, then no quantum excitation effects are included,
  which is to say that all particles will experience the same perturbation.  
\item \verb|LOSSES| --- Default is 1.  If set to 0, no average energy losses are included.
\end{itemize}
  

{\em There are a number of caveats that must be observed when using this element.}

\begin{enumerate}

\item If there is dispersion at the location of the \verb|SREFFECTS| element,
the closed orbit will change because of the average momentum change, but it will disagree with
tracking results. The reason is that in tracking \verb|SREFFECTS| must displace the beam to the new
equilibrium orbit, because otherwise there will be additional betatron motion excited
and the wrong equilibrium emittance will be obtained.  
(Since the \verb|SREFFECTS| element is already adding the betatron motion excitation for the
entire ring, {\tt elegant} is forced to offset each particle by $\Delta \delta\vec{\eta}$
to suppress any additional excitation.)

This issue can be resolved by placing the \verb|SREFFECTS| element next to the
rf cavity and setting \verb|INCLUDE_OFFSETS=0|.  Since the average momentum change is
zero from the two elements, no additional betatron motion will be generated.
Optionally, one can also use many \verb|SREFFECTS| elements at equivalent locations in
the lattice, which will decrease the magnitude of the effect.

\item When used for dynamic aperture and momentum aperture determination,  
one should set \verb|QEXCITATION=0|.  Putting the rf cavity (if any) right next to
the \verb|SREFFECTS| element is a good idea to avoid spurious excitation of betatron
motion.

\item Nothing prevents including this element in a lattice when doing frequency map analysis, although
it probably doesn't make any sense.  Only the average energy loss per turn will be included.
Again, putting an rf cavity right after \verb|SREFFECTS| is a good idea.

\item In versions 19.0 and later, {\tt elegant} includes the effect of {\tt
SREFFECTS} on the closed orbit.  This presents a dilemna when
automatic initialization is used, because in order to perform
automatic initialization, {\tt elegant} has to compute the optics
functions.  However, it must determine the closed orbit to compute the
optics functions.  The solution to this is for the user to pre-compute
the twiss parameters and radiation integrals using \verb|twiss_output|
with \verb|output_at_each_step=0|.  The user is free to subsequently
give \verb|twiss_output| with \verb|output_at_each_step=1| to obtain
the results on the closed orbit.

\item Computation of Twiss parameters does not fully include the
effects of synchrotron radiation losses when these are imposed using
{\tt SREFFECTS} elements.  If {\tt PREF=0} (automatic initialization),
these effects are completely missing.  If {\tt PREF} is non-zero, then
{\tt elegant} will use the {\tt DDELTAREF} parameter to compute the
energy offset from the element, and thus its effect on the beam
trajectory.

\end{enumerate}
