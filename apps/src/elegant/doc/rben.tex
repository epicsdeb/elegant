When adding errors, care should be taken to choose the right
parameters.  The \verb|FSE| and \verb|ETILT| parameters are used for
assigning errors to the strength and alignment relative to the ideal
values given by \verb|ANGLE| and \verb|TILT|.  One can also assign 
errors to \verb|ANGLE| and \verb|TILT|, but this has a different meaning:
in this case, one is assigning errors to the survey itself.  The reference
beam path changes, so there is no orbit/trajectory error. The most common
thing is to assign errors to \verb|FSE| and \verb|ETILT|.  Note that when
adding errors to \verb|FSE|, the error is assumed to come from the power
supply, which means that multipole strengths also change.

{\em Special note about splitting dipoles}: when dipoles are long, it is
common to want to split them into several pieces, to get a better look
at the interior optics.  When doing this, care must be exercised not
to change the optics.  {\tt elegant} has some special features that
are designed to reduce or manage potential problems. At issue is the
need to turn off edge effects between the portions of the same dipole.

First, one can simply use the \verb|divide_elements| command to set up
the splitting.  Using this command, {\tt elegant} takes care of everything.

Second, one can use a series of dipoles {\em with the same name}.  In this case,
elegant automatically turns off interior edge effects.  This is true when the
dipole elements directly follow one another or are separated by a MARK element.

Third, one can use a series of dipoles with different names.  In this case, you
must also use the \verb|EDGE1_EFFECTS| and \verb|EDGE2_EFFECTS| parameters to
turn off interior edge effects.  

