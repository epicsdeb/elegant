{\bf NB:} Although this element is correct insofar as it uses the
fields for a pure TM110 mode, it is recommended that the {\tt RFDF}
element be used instead.  In a real deflecting cavity with entrance
and exit tubes, the deflecting mode is a hybrid TE/TM mode, in which
the deflection has no dependence on the radial coordinate.

To derive the field expansion, we start with some results from
Jackson\cite{Jackson}, section 8.7.  The longitudinal electric field
for a TM mode is just
\begin{equation}
E_z = - 2 i E_0 \Psi(\rho, \phi) \cos \left(\frac{p \pi z}{d}\right) e^{-i\omega t},
\end{equation}
where $p$ is an integer, $d$ is the length of the cavity, and we use
cylindrical coordinates $(\rho, \phi, z)$.  The factor of $-2i$ represents a
choice of sign and phase convention.  We are interested in the
TM110 mode, so we set $p=0$.  In this case, we have
\begin{equation}
E_x = E_y = 0 
\end{equation}
and (using CGS units)
\begin{equation}
\vec{H} = - 2 i E_0 \frac{i \epsilon \omega}{c k^2} \hat{z} \times \nabla \Psi e^{-i \omega t}.
\end{equation}
For a cylindrical cavity, the function $\Psi$ for the $m=1$ aximuthal mode is 
\begin{equation}
\Psi(\rho, \phi) = J_1 (k \rho) \cos \phi,
\end{equation}
where $k = x_{11}/R$, $x_{11}$ is the first zero of $J_1(x)$, and $R$ is the cavity radius.
We don't need to know the cavity radius, since $k = \omega/c$, where $\omega$ is the
resonant frequency.  By choosing $\cos\phi$ for the aximuthal dependence, we'll get 
a magnetic field primarily in the vertical direction.

In MKS units, the magnetic field is
\begin{equation}
\vec{B} = \frac{2 E_0}{k c} e^{-i \omega t} \left( \hat{\rho} \frac{J_1(k\rho)}{\rho} \sin \phi
        + \hat{\phi} \cos\phi \frac{\partial J_1(k\rho)}{\partial \rho}\right).
\end{equation}

Using {\tt mathematica}, we expanded these expressions to sixth order
in $k*\rho$.  Here, we present only the expressions to second
order. Taking the real parts only, we now have
\begin{eqnarray}
E_z & \approx & E_0 k \rho \cos \phi \sin \omega t \\
c B_\rho  & \approx & E_0 \left(1 - \frac{k^2 \rho^2}{8}\right)\sin\phi \cos\omega t \\
c B_\phi  & \approx & E_0 \left(1 - \frac{3 k^2 \rho^2}{8}\right)\cos\phi \cos\omega t 
\end{eqnarray}
The Cartesian components of $\vec{B}$ can be computed easily
\begin{eqnarray}
c B_x & = & c B_\rho\cos\phi - c B_\phi\sin\phi \\
      & = & \frac{E_0}{4} \rho^2 k^2 \cos\phi \sin\phi \cos\omega t \\
c B_y & = & c B_\rho\sin\phi + c B_\phi\cos\phi \\
      & = & E_0 \left(1 - \frac{k^2\rho^2 (2 \cos^2\phi + 1)}{8}\right) \cos\omega t 
\end{eqnarray}

The Lorentz force on an electron is $F = -e E_z \hat{z} - e c \vec{\beta} \times \vec{B}$,
giving
\begin{eqnarray}
F_x/e & = & \beta_z c B_y \\
F_y/e & = & -\beta_z c B_x \\
F_z/e & = & -E_z - \beta_x c B_y + \beta_y c B_x 
\end{eqnarray}
We see that for $\rho \rightarrow 0$, we have $E_z = 0$, $B_x = 0$, and
\begin{equation}
c B_y = E_0 \cos \omega t.
\end{equation}
Hence, for $\omega t=0$ and $E_0>0$ we have $F_x>0$.  This explains
our choice of sign and phase convention above.  Indeed, owing to the
factor of $2$, we have a peak deflection of $e E_0 L/E$, where $L$ is
the cavity length and $E$ the beam energy.  Thus, if $V = E_0 L$ is
specified in volts, and the beam energy expressed in electron volts,
the deflection is simply the ratio of the two.  As a result, we've
chosen to parametrize the deflection strength simply by referring to
the ``deflecting voltage,'' $V$.
