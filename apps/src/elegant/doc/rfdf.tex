This cavity provides a transverse deflection that is constant as
a function of transverse coordinates.  It is probably the best model
for a real cavity, because real cavities contain a mixture of TM- and
TE-like modes that result in a uniform deflection.

For simplicity of use, the deflection is specified as a voltage, even 
though it originates in a magnetic field.  The magnetic field is
\begin{equation}
B = B_0 \hat{y} \cos \omega t
\end{equation}
The corresponding electric field is obtained from Faraday's law (MKS units)
\begin{equation}
\left(\nabla \times \vec{E}\right)_y = - \frac{\partial \vec{B}}{\partial y}.
\end{equation}
Assuming $E_x = E_y = 0$, we have
\begin{equation}
E_z = B_0 \omega x \sin \omega t.
\end{equation}

The change in momenta (in units of $m c$) in passing through a slice of length $\Delta L$ is
\begin{eqnarray}
\Delta p_x & = & \frac{q B_0 \Delta L}{m c} \cos \omega t \\
\Delta p_y & = & 0 \\
\Delta p_z & = & \frac{q B_0 \omega x \Delta L}{m c^2} \sin\omega t 
\end{eqnarray}

If we want to think in terms of a deflecting voltage, we can re-write this as
\begin{eqnarray}
\Delta p_x & = & \frac{q V}{m c^2} \cos \omega t \\
\Delta p_y & = & 0 \\
\Delta p_z & = & \frac{q V}{m c^2} k x \sin\omega t,
\end{eqnarray}
where $k = \omega/c$.

