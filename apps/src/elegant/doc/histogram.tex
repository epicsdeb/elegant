The output filename may be an incomplete filename.  In the case of the
\verb|HISTOGRAM| point element, this means it may contain one instance of
the string format specification ``\%s'' and one occurence of an
integer format specification (e.g., ``\%ld'').  {\tt elegant} will
replace the format with the rootname (see
\verb|run_setup|) and the latter with the element's occurrence
number.  For example, suppose you had a repetitive lattice defined as
follows:
\begin{verbatim}
H1: HISTOGRAM,FILENAME=''%s-%03ld.h1''
Q1: QUAD,L=0.1,K1=1
D: DRIFT,L=1
Q2: QUAD,L=0.1,K1=-1
CELL: LINE=(H1,Q1,D,2*Q2,D,Q1)
BL: LINE=(100*CELL)
\end{verbatim}
The element \verb|H1| appears 100 times.  Each instance will result in
a new file being produced.  Successive instances have names like
``{\em rootname}-001.h1'', ``{\em rootname}-002.h1'', ``{\em
rootname}-003.h1'', and so on up to ``{\em rootname}-100.h1''.  (If
instead of ``\%03ld'' you used ``\%ld'', the names would be ``{\em
rootname}-1.h1'', ``{\em rootname}-2.h1'', etc. up to ``{\em
rootname}-100.h1''.  This is generally not as convenient as the names
don't sort into occurrence order.)

The files can easily be plotted together, as in 
\begin{verbatim}
% sddsplot -column=dt,dtFrequency *-???.h1 -separate 
\end{verbatim}
They may also be combined into a single file, as in
\begin{verbatim}
% sddscombine *-???.h1 all.h1 
\end{verbatim}

In passing, note that if \verb|H1| was defined as
\begin{verbatim}
H1: HISTOGRAM,FILENAME=''%s.h1''
\end{verbatim}
or 
\begin{verbatim}
H1: HISTOGRAM,FILENAME=''output.h1''
\end{verbatim}
only a single file would be produced, containing output from the last instance
only.

