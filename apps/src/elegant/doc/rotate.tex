The sign convention for the {\tt TILT}  parameter is confusing on this element.
In particular, a positive {\tt TILT} rotates the beam counter-clockwise about the
longitudinal axis.  This is the opposite sense to rotations of elements, where a 
positive {\tt TILT} rotates the element clockwise about the longitudinal axis.

Hence, if one wanted to rotate a series of elements by 0.1 rad, one could do the
following:
\begin{verbatim}
ROT1: ROTATE,TILT=0.1
ROT2: ROTATE,TILT=-0.1
BL: line=(ROT1,...,ROT2)
\end{verbatim}
The {\tt TILT} value for {\tt ROT1} is the same (including the sign) as the individual
{\tt TILT} values one would give to all the elements represented by \verb|...|.

