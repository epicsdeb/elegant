This element provides a matrix-based implementation of an alpha magnet \cite{Enge}. 
Matrices up to third order are available \cite{Borland_thesis}.

The parameter {\tt XMAX} determines the size of the alpha, which is related
to the gradient $g$ in the magnet and the central momentum $\beta\gamma$ by
\begin{equation}
x_{max} [m] = 0.07504986 \sqrt{\frac{\beta\gamma}{g [T/m]}}.
\end{equation}
The path length of the central particle is $2.554 x_{max}$.

Because an alpha magnet has large dispersion at the midplane, it is often used
for momentum filtration in addition to bunch compression.  The dispersion at the
center is given by the simple relation 
\begin{equation}
R_{16} = -\frac{1}{2} x_{max}.
\end{equation}
To use an alpha magnet for momentum filtration in {\tt elegant}, one must split the
alpha magnet into two pieces.  One may then either use the scraper features of the
{\tt ALPH} element or other elements such as {\tt SCRAPER} or {\tt RCOL}.

To split an alpha magnet, one uses the {\tt PART} parameter.  E.g.,
\begin{verbatim}
! First half, with momentum filter between -5% and +2.5%
AL1: ALPH,XMAX=0.11,PART=1,DP1=-0.05,DP2=0.025
! Second half 
AL2: ALPH,XMAX=0.11,PART=2
AL:  LINE=(AL1,AL2)
\end{verbatim}

As just illustrated, the parameters {\tt DP1} and {\tt DP2} may be used to filter the
momentum by providing fractional momentum deviation limits.  These are implemented in
a physical fashion by computing the corresopnding horizontal position deviations and
imposing these as limits on the particle coordinates.  One may also do this directly
using the {\tt XS1} and {\tt XS2} parameters, which specify maximum acceptable deviations
from the nominal horizontal position.  {\tt XS1} is the allowed deviation on the low-energy
side while {\tt XS2} is the allowed deviation on the high-energy side.
