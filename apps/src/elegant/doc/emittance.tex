This element allows changing the  emittance of a beam during tracking.
It  is intended to  be used  to modify  the emittance  ``slightly'' to
agree  with,  for   example,  experimental  measurements.   

The LCLS provides an example application: we track a beam from a
photo-injector simulation through a laser/undulator beam heater and
then through the entire linac.  The beam emittance and twiss
parameters are measured at a diagnostic downstream of the laser
heater.  We can insert an EMITTANCE element and a TWISS element at the
location of the diagnostic to force the beam properties to the exact
values that are measured.  This compensates for imperfect modeling of
the photo-injector while allowing us to conveniently model the system
between the photo-injector and the point at which the emittance is
measured.

