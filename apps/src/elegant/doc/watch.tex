The output filename may be an incomplete filename.  In the case of the
\verb|WATCH| point element, this means it may contain one instance of
the string format specification ``\%s'' and one occurence of an
integer format specification (e.g., ``\%ld'').  {\tt elegant} will
replace the format with the rootname (see
\verb|run_setup|) and the latter with the element's occurrence
number.  For example, suppose you had a repetitive lattice defined as
follows:
\begin{verbatim}
W1: WATCH,FILENAME=''%s-%03ld.w1''
Q1: QUAD,L=0.1,K1=1
D: DRIFT,L=1
Q2: QUAD,L=0.1,K1=-1
CELL: LINE=(W1,Q1,D,2*Q2,D,Q1)
BL: LINE=(100*CELL)
\end{verbatim}
The element \verb|W1| appears 100 times.  Each instance will result in
a new file being produced.  Successive instances have names like
``{\em rootname}-001.w1'', ``{\em rootname}-002.w1'', ``{\em
rootname}-003.w1'', and so on up to ``{\em rootname}-100.w1''.  (If
instead of ``\%03ld'' you used ``\%ld'', the names would be ``{\em
rootname}-1.w1'', ``{\em rootname}-2.w1'', etc. up to ``{\em
rootname}-100.w1''.  This is generally not as convenient as the names
don't sort into occurrence order.)

The files can easily be plotted together, as in 
\begin{verbatim}
% sddsplot -column=t,p *-???.w1 -graph=dot -separate 
\end{verbatim}
They may also be combined into a single file, as in
\begin{verbatim}
% sddscombine *-???.w1 all.w1 
\end{verbatim}

In passing, note that if \verb|W1| was defined as
\begin{verbatim}
W1: WATCH,FILENAME=''%s.w1''
\end{verbatim}
or 
\begin{verbatim}
W1: WATCH,FILENAME=''output.w1''
\end{verbatim}
only a single file would be produced, containing output from the last instance
only.

N.B.: confusion sometimes occurs about some of the quantities related
to the {\tt s} coordinate in this file when in parameter mode.  Please
see Section \ref{sec:longitCoord} above.
